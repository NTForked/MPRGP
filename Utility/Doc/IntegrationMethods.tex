\documentclass[twocolumn]{article}
\usepackage {bibentry}
\setlength{\columnsep}{3em}
\author{Siwang Li}
\title{Integration Methods In Elastic Object Simulation}

\begin{document}
\maketitle

\section{Abstract}
In this document, we introduce the basic integration algorithms for motion
equations, as well as the methods for implementing the position constraints. 

\section{Motion Equation}
The simulation of elastic objects can be described using the following ODEs 
\begin{equation} \label{motion_eq}
  M\ddot{q} + D(q)\dot{q} + f(q) = f^{ext}
\end{equation}
And the position constraints usually can be expressed as
\begin{equation} \label{postion_con}
{C} q = u_c
\end{equation}
Here ${C}$ is the constraint matrix, and $u_c$ is the displacements of the
constrained nodes.

\section{No Constraints}
To simplify the introduction, firstly, we introduce the algorithms with no
constraints considered.
\subsection{Implicit Integration} \label{sec:implicit-integration}
First, we rewrite equation (\ref{motion_eq}) in the form of 
\[
\left\{ \begin{array}{rl}
    M\dot{v} &= f^{ext}- D(q)v - f(q)\\
    \dot{q} &= v
  \end{array} \right.\nonumber
\]
And at time step $k+1$, we compute the status $(q_{k+1},v_{k+1})$ by solving 
\[
\left\{ \begin{array}{rl}
    Mv _{k+1}  &= Mv _{k} +h(f^{ext}_{k+1} - D(q _{k+1} )v _{k+1}  - f(q _{k+1} ))\\
    {q} _{k+1} &= q _{k} + hv _{k+1} 
  \end{array} \right.\nonumber
\]
We apply the first order Taylor approximation to $f (q_{k+1}) $ at $q_k$
\[
f(q _{k+1}) = f(q_k + hv _{k+1} ) \approx f(q_k) + K(q_k)(hv _{k+1})
\]
and simply replace $D(q _{k+1} )$ with $D(q _{k} )$, then we get
\[
\left\{ \begin{array}{rl}
    (M+hD(q _{k}) + h^2K(q_k))v _{k+1}  &= Mv _{k} +h(f^{ext}_{k+1} - f(q _{k} ))\\
    {q} _{k+1} &= q _{k} + hv _{k+1} 
  \end{array} \right.\nonumber
\]
e.g.
\[
\left\{ \begin{array}{rl}
    A_k v _{k+1}  &= b_k\\
    {q} _{k+1} &= q _{k} + hv _{k+1} 
  \end{array} \right.\nonumber
\]
where 
\[
\left\{ \begin{array}{rl}
    A_k &= (1+h\alpha_m)M + h(h+\alpha_k)K(q_k) \\
    b_k &= Mv _{k} +h(f^{ext}_{k+1} - f(q _{k} ))
  \end{array} \right.\nonumber
\]
\subsection{Semi-implicit Integration}
At time step $k+1$, we compute the status $(q_{k+1},v_{k+1})$ by solving 
\[
\left\{ \begin{array}{rl}
    Mv _{k+1}  &= Mv _{k} +h(f^{ext}_{k} - D(q _{k} )v _{k}  - f(q _{k} ))\\
    {q} _{k+1} &= q _{k} + hv _{k+1} 
  \end{array} \right.\nonumber
\]

\subsection{Explicit Integration}
At time step $k+1$, we compute the status $(q_{k+1},v_{k+1})$ by solving 
\[
\left\{ \begin{array}{rl}
    Mv _{k+1}  &= Mv _{k} +h(f^{ext}_{k} - D(q _{k} )v _{k}  - f(q _{k} ))\\
    {q} _{k+1} &= q _{k} + hv _{k} 
  \end{array} \right.\nonumber
\]

\section{Position Constraints}
There are two common approaches for implement the position constraints, as
described bellow.
\subsection{Penalty Method}
In this method, we trait the position constraint (\ref{postion_con}) as a
penalty term to minimize
\[
E_c(q) = \frac{1}{2}\lambda_c \|Cq-u_c\|^2_2
\]
and we apply the gradient of this term as a type of penalty force to the
original motion equation (\ref{motion_eq}), then we have
\begin{equation} \label{motion_eq_con}
   M\ddot{q} + D(q)\dot{q} + f(q) + (\frac{\partial{E_c(q)}}{\partial{q}})^T=
   f^{ext}
\end{equation}
where 
\[
(\frac{\partial{E_c(q)}}{\partial{q}})^T = (\lambda_c(Cq-u_c)^TC)^T =
\lambda_cC^TC q - \lambda_c C^Tu_c
\]
Then equation (\ref{motion_eq_con}) can be written as 
\[
  M\ddot{q} + D(q)\dot{q} +  \tilde{f}(q) = \tilde{f} ^{ext} 
\]
where
\[
\left\{ \begin{array}{rl}
    \tilde{f}^{ext} &= {f}^{ext} + \lambda_cC^Tu_c \\
    \tilde{f}(q) &= f(q)+\lambda_cC^TCq
  \end{array} \right.\nonumber
\]

\subsubsection{Implicit Integration}
As the constrained equation we mentioned previous, the integration for this
equation is similar to the ones with no constraints considered. And as
\begin{eqnarray}
\tilde{f}(q _{k+1} ) &=&  f(q _{k+1}) + \lambda_cC^TC q _{k+1} \nonumber \\
&=& f(q _{k} + hv _{k+1} ) + \lambda_cC^TC (q _{k} + hv _{k+1}) \nonumber \\
&\approx&  h(K(q_k)+\lambda_c C^TC )v _{k+1} + (f(q_k)+\lambda_cC^TCq_k) \nonumber
\end{eqnarray}
then, when implicit strategy is used to solve equation (\ref{motion_eq_con}),
the integration algorithm described in section \ref{sec:implicit-integration}
should only be slightly modified by setting
\[
\left\{ \begin{array}{rl}
    \tilde{A}_k &= A_k + h^2\lambda_cC^TC\\
    \tilde{b}_k &= b_k + h\lambda_c(C^Tu_c - C^TCq_k)
  \end{array} \right.\nonumber
\]

\subsection{Lagrange Method}
With Lagrange constrained mechanics, the motion equation \ref{motion_eq} with
constraints (\ref{postion_con}) can be stated as 
\begin{equation} \label{motion_eq_lag_1} 
  \left\{ \begin{array}
      {rl} M\ddot{q} + D(q)\dot{q} + f(q) + (\frac{\partial{(Cq -
          u_c)}}{\partial{q}})^T\lambda &= f^{ext} \\
      Cq &= u_c
    \end{array} \right.
\end{equation}
e.g
\[
\left\{ \begin{array}{rl}
        M\ddot{q} + D(q)\dot{q} + f(q) + C^T\lambda &= f^{ext} \\
        Cq &= u_c
  \end{array} \right.\nonumber
\]
Where $\lambda$ is the Lagrange multipliers. This equation can be solved by
assemble $\lambda$ as part of the motion equation 
\[
    \left[ \begin{array}{c}
        M\ddot{q} \\
        O
      \end{array} \right] + 
    \left[ \begin{array}{c}
        D(q)\dot{q}\\
        O
      \end{array} \right] +
    \left[ \begin{array}{c}
        f(q) \\
        Cq
      \end{array} \right] + 
    \left[ \begin{array}{c}
        C^T\lambda \\
        O
      \end{array} \right] =
    \left[ \begin{array}{c}
        f^{ext}  \\
        u_c
      \end{array} \right]
\]
Yet, for conveniently introduction, we introduce the constraints during the
integration process directly. 
\subsubsection{implicit integration}
\[
\left\{ \begin{array}{rl}
    A_k v _{k+1}  &= b_k\\
    {q} _{k+1} &= q _{k} + hv _{k+1} \\
  \end{array} \right.\nonumber
\]
subject to
\[
C q_{k+1} = u_c
\]
Similar to (\ref{motion_eq_lag_1}),with Lagrange constrain mechanics
\[
\left\{ \begin{array}{rl}
    A_k v _{k+1} + C^T\lambda &= b_k\\
    C (q_{k}+ hv _{k+1}) &= u_c \\
    {q} _{k+1}  &= q _{k}  + hv _{k+1}
  \end{array} \right.\nonumber
\]

\[
\left\{ \begin{array}{rl}
    \tilde{A}_k \tilde{v} _{k+1}  &= \tilde{b}_k\\
    {q} _{k+1}  &= q _{k}  + hv _{k+1}
  \end{array} \right.\nonumber
\]
where
\[
    \tilde{A}_k = 
    \left[ \begin{array}{cc}
        A_k & C^T\\
        C & O
      \end{array} \right]
\]
\[
    \tilde{b}_k =     
    \left[ \begin{array}{c}
        b_k\\
        \frac{1}{h}(u_c - Cq_k)
      \end{array} \right]    
\]
\[
    \tilde{v}_{k+1} = 
    \left[ \begin{array}{c}
        v_{k+1} \\
        \lambda
      \end{array} \right] 
\]
It should be note here that, though $\tilde{A}_k$ is symmetric, it's not
positive definite, so it must be solved using $LU$ approach. And other method
such as $LLT$ and $LDLT$ will result in unpleasing results.
\end{document}
